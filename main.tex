\documentclass{article}
\usepackage{graphicx}
\usepackage{fancyhdr}
\usepackage{color}
\usepackage{hyperref}
\usepackage[margin=1in]{geometry}
\usepackage[affil-it]{authblk} 
\usepackage{etoolbox}
\usepackage{lmodern}
\usepackage[fontsize=12pt]{fontsize}
\usepackage{setspace}
\usepackage[backend=biber,
style=numeric]{biblatex}
\usepackage{tabularx}
\usepackage{array}
\usepackage{longtable}
\usepackage{subfig}
\usepackage{float}
\usepackage{amsmath,mathtools}
\usepackage{graphicx} % Required for inserting images

\usepackage{listings}
\usepackage{xcolor}

\lstset{%
  language=Matlab,
  basicstyle=\ttfamily\small,
  keywordstyle=\color{black}\bfseries,
  commentstyle=\color{gray},
  stringstyle=\color{black},
  numbers=left,
  numberstyle=\tiny\color{gray},
  stepnumber=1,
  numbersep=5pt,
  frame=single,
  breaklines=true,
  tabsize=4
}


\title{DBS Extra Credit | Question 4 Part B }

\author{Kaeshev Alapati}
\date{Feb 2026}

\begin{document}

\maketitle

\maketitle
\thispagestyle{empty}
\pagestyle{fancy}
\fancyhead{}
\fancyfoot{}

\pagebreak

\tableofcontents

\pagebreak
\onehalfspacing

\setcounter{page}{1}

\rhead{Page \thepage}


\section{Question 4 part (b)}
You are given the following set F of functional dependencies for a relation R(A,B,C,D,E).
\begin{align}
    F= \{\\
    AB \longrightarrow CE,\\
    A \longrightarrow D,\\
    BE \longrightarrow D,\\
    CDE \longrightarrow AB,\\
    BC \longrightarrow DE\}
\end{align}

\subsection{b (Problem)}

If we remove $AB \longrightarrow E$ from $\mathcal{F}$ does it change the closure of $\mathcal{F}$ (i.e. $\mathcal{F}^+$?

To show this check wether $\mathcal{F} - \{AB \longrightarrow E\}$ implies $AB \longrightarrow E$. You can either use inference rules or demonstrate $\{AB\}^+$ in $\mathcal{F} - \{AB \longrightarrow E\{$.

\subsection{b (Solution)}

F becomes
\begin{align}
    F' = \{\\
    AB \longrightarrow C,\\
    A \longrightarrow D,\\
    BE \longrightarrow D,\\
    CDE \longrightarrow AB,\\
    BC \longrightarrow DE\}
\end{align}

The most simple way to demonstrate if the removal changes $\mathcal{F}^+$ is to demonstrate $\{AB\}^+$.

\begin{center}
\begin{tabular}{ c c }
 \textbf{Output} & \textbf{Rule} \\
 $AB$ & beginning \\
 $ABC$ & $AB \longrightarrow C$ \\
 $ABCD$ & $A \longrightarrow D$ \\ 
 $ABCDE$ & $BC \longrightarrow DE$.
\end{tabular}
\end{center}

As the closure is $ABCDE$ (it encloses E) the removal does \textbf{not} change $\mathcal{F}^+$. The same input AB can achieve the same output CE. Even if the rule $AB \longrightarrow E$ is removed.


Additionally this property can be demonstrated via inference rules demonstrated below.

\begin{center}
\begin{tabular}{ c c }
 \textbf{Output} & \textbf{Rule} \\
 $AB \longrightarrow C$ (1) & Given \\
 $BC \longrightarrow DE$ (2) & Given \\
 $AB \longrightarrow BC$ (3) & Reflexivity of B in (1) \\ 
 $AB \longrightarrow DE$ (4) & Transitivity of (3) and (2). \\
 $AB \longrightarrow E$ (5) & Decomposition of (4).
\end{tabular}
\end{center}

We can see in (5) that using inference rules $AB \longrightarrow E$ can be recovered from $\mathcal{F} - \{AB \longrightarrow E\{$. This means that removing $AB \longrightarrow E$ does \textbf{not} change $\mathcal{F}^+$.
\end{document}
